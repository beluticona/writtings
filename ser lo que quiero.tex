\documentclass[12t, a4 paper]{article}


\usepackage{lmodern}
\usepackage[T1]{fontenc}
\usepackage[utf8]{inputenc} 
\usepackage[spanish]{babel}



\title{¿Por qu\'e no somos lo que queremos ser?}
\date{}

\begin{document}

\maketitle

\begin{flushright}
-Todos tenemos una historia que compartir. \\
-No todos, yo no creo tener una. \\
-¿No? Seguro ten\'es una, solo es cuesti\'on de encontrarla...¿por qu\'e quisiste ser profesor?\\
 -Yo no quer\'ia ser profesor, esa es una historia un poco triste...\\
 \end{flushright}
\rightline{ \textit{-An\'onimo}}

\textbf{En alguna tarde del 2015...} \\

Cuando le escuch\'e decir “quer\'ia” me qued\'e pensando en todas esas personas que no viven de su vocaci\'on, que no tuvieron la posibilidad de seguir aquello que les hubiera encantado. Ese, es el caso de mis padres (y segura y tristemente much\'isimos m\'as): vivir todos los d\'ias conduciendo un auto con el ruido del tr\'afico diario no debi\'o ser su sue\~no de joven, pero es lo que le toc\'o en la vida...

La vida nos lleva por m\'ultiples caminos: cada d\'ia vemos abrirse y cerrarse un mont\'on de puertas, re abrirse y re cerrarse al d\'ia siguiente, y as\'i sucesivamente.

	\begin{quote} 
	¡Árbol de la vida!, ten piedad y sigue ramific\'andote que a\'un no he encontrado mi camino.
	\end{quote} 
	\rightline{ \textit{-B.T.}}

\vspace{2mm}
Encontrar ese camino debe ser la gracia de la vida. A veces es m\'as injusta con algunos y les impone bastas responsabilidades de j\'ovenes, muri\'endose poco a poco as\'i sus chances de elegir su futuro. Otros tienen la posibilidad y a\'un as\'i, al no poder apreciar la dicha que tienen, dejan pasar el tiempo hasta que en cierto punto se preguntan qu\'e hacer con su vida. Darle sentido no es f\'acil. Otros prefieren simplemente vivirla porque as\'i como un d\'ia est\'as aqu\'i, al otro, ya no. En esto \'ultimo pens\'e cuando le dije: “todos tenemos una historia que compartir”. Que esa historia, ese mensaje, ese aprendizaje de vida, esa huella que tal vez sea lo m\'as rico -humanamente hablando- que podamos dejar, se suela reci\'en apreciar en los momentos m\'as cr\'iticos de nuestras vidas, realmente deja mucho en qu\'e pensar y reflexionar.

Es el mismo camino el que nos ense\~na cruelmente y sin anastesia. Seguramente por eso los mayores son los que m\'as ense\~nanzas siempre han tenido para transmitir. Pero, ¿acaso se debe llegar a cierta edad para poder apreciar lo bello que ha sido el camino? ¿c\'omo ha llenado nuestras p\'aginas de tantas alegr\'ias y tristezas? Yo creo que no solo se crece con los a\~nos, sino tambi\'en a partir de las experiencias de vida de los que nos rodean. Saber escucharlas es el secreto.

La familia es de donde suele venir la primera lecci\'on de la vie.

Desde chica he aprendido que a pesar de lo que nos toca en la vida, es uno el que decide el rumbo de la misma. Nacemos con un vector director definido, pero no existen l\'imites dentro de ese plano para nuestra imaginaci\'on. Para m\'i una de las clases de personas que m\'as admiro son aquellas que han trabajado, estudiado y criado a sus hijos para sacar a la familia adelante. Cuando existe una causa honesta y sincera, como darle lo mejor que puedas a tus hijos, es ah\'i cuando se puede ver el potencial que uno puede llegar a tener. Mi madre es una de esa clase de personas y es gracias a ella que he aprendido de chica que es solo a trav\'es de la educaci\'on que se puede progresar. Recuerdo a la perfecci\'on una clase de mi profesor Lorenz en la que \'el comentaba c\'omo para muchas generaciones de inmigrantes la educaci\'on de sus hijos lo era todo. 

Cada d\'ia trato de valorar todo lo que tengo: cada cosa por las que ellos han trabajado, cada cosa por la que no durmieron, no comieron o cada camino al que renunciaron por nosotros. Hubo otros tiempos en los que quise dedicarles todo: si una buena calificaci\'on los tranquilizaba, entonces daba todo para obtenerla. Si hab\'ia algo que pod\'ia hacer (lo que sea), lo hac\'ia. 

Reconfortar a aquellos quienes te han dado una mano en la vida debe ser una de las cosas m\'as lindas que, al menos yo, pueda sentir. El punto de quiebre est\'a cuando uno se haya haciendo cosas por otros, haciendo su vida por otras personas y no por s\'i mismo. A pesar de todo lo que ha tenido que suceder para que yo est\'e aqu\'i escribiendo, llegu\'e a la conclusi\'on de que antes de poner a los otros delante de mi vida, antes de tratar de alegrar a otros, debo ser yo la que debe estar feliz consigo misma. No hay manera de ayudar a otro sin estar uno feliz consigo mismo. En vez de darle una “buena” nota a mis padres, deb\'ia darme la felicidad a m\'i para que ellos puedan ver que todo lo que hacen ha valido la pena. Vali\'o la pena :’) 

Encontrar la felicidad para m\'i es encontrarse a s\'i mismo. Camino dif\'icil y desafiante, el cual  estoy segura que deja una historia para compartir. Hace no tan poco vi una charla TED\footnote{Disponible en: http://tedxtalks.ted.com/video/El-poder-de-una-conversacin-Alv}  que trasmit\'ia este mismo mensaje. El tren pasa solo una vez y es f\'acil arrepentirse de lo hecho, pero imposible de lo jam\'as realizado. Si no te gusta tu trabajo, sal para encontrar aquel que s\'i. Sal a estudiar lo que siempre quisiste, nunca es tarde para aprender y nunca es tarde para volver a ser chicos y so\~nar con tanta facilidad. Nunca ser\'a tarde para salir a buscarlo, para salir a buscarse a s\'i mismo, pero s\'i puede que llegue el d\'ia que sea tarde para escuchar o compartirte con el resto. Todo lo que no compartas en tu vida, morir\'a contigo. Entonces, ¿para qu\'e esperar al ma\~nana? ¿Para qu\'e esperar a ese cruce con el m\'as all\'a para compartirte, para compartir tu historia? 

\vspace{2mm}

\textsc{Me arrepentir\'e de lo que hable pero jam\'as de lo que calle.}

\end{document}