\documentclass[12t, a4 paper]{article}


\usepackage{lmodern}
\usepackage[T1]{fontenc}
\usepackage[utf8]{inputenc} 
\usepackage[spanish]{babel}


\title{Con hambre no se puede pensar }
\date{}
\begin{document}
\maketitle
\textbf{Post OAQ 2013...} \\

Tras el certamen nacional OAQ 2013 volví reflexionando sobre mi vida, la primaria, mis compañeros, mis viejos...A las semanas se hizo la primer jornada de Educando al Cerebro en el CNBA y las charlas de Ballarini y Furman fueron las que me llegaron de verdad. Ambas experiencias despertaron algo en mí que todavía hoy trato de comprender. Le pido disculpas al profesor o educador a quien le quité el lugar porque el Aula Magna estaba repleta y muchos no pudieron entrar. 

    Saber que aporté con un mínimo voto para que ganen una beca que les permita seguir haciendo las jornadas a largo del país realmente me pone feliz. Saber que muchísimos profesores y educadores van a conocer nuevas metodologías de enseñanza para motivar y despertar curiosidad por aprender a sus alumnos, me motiva día a día. Saber que en el día de mañana se va depositar una semillita de curiosidad en muchísimos chicos (o mejor aún, no se va a estructurar o apagar su curiosidad innata), que se va a potenciar sus ganas de explorar lo desconocido, sin importar su situación económica, sin importar su condición social, sin importar su religión, su identidad de género, su condición física o nacionalidad...ESO me alegra la vida. Porque cuando uno tiene un conocimiento, si no lo comparte, muere en uno. Porque para acortar esa brecha tenemos que darle una mano a los que realmente no pueden tener acceso a lo que por ahí vos sí. 

    Saber valorar y apreciar las oportunidades, lo que nos rodea cada día si tenemos la dicha, es un primer paso. Compartirlo con el de al lado, es el siguiente.  \\

 
  \textbf{ Cuando decimos que más ciencia es más libertad, lo decimos en serio. }
   
\end{document}