\documentclass[16pt, a4 paper]{article}


\usepackage{lmodern}
\usepackage[T1]{fontenc}
\usepackage[utf8]{inputenc} 
\usepackage[spanish]{babel}




\title{¿Qu\'e es del arte en la ciencia? ¿Qu\'e es de la ciencia en el arte?}
\date{}

\begin{document}

\maketitle

\textbf{Diciembre 2015, terminando 6to año...} \\

Lo que comenz\'o como una linda experiencia en laboratorio, como una posibilidad de aprender jugando a preparar cosas raras, soluciones de llamativos colores y piedras de m\'ultiples tama\~nos y formas, termin\'o en la construcci\'on de un puente entre dos cosas generalmente concebidas como de mundos ajenos. 

La respuesta a la primera interrogante la hall\'e el a\~no pasado participando en un concurso de crecimiento de cristales. Con un grupo de amigas hicimos mariposas azules de $CuSO_{4}$ (sulfato de cobre) y tratamos de memorar as\'i, a nuestro modo, el A\~no Internacional de la Cristalograf\'ia y el A\~no Cortazariano por el cent\'esimo aniversario de su nacimiento. M\'as all\'a de los resultados obtenidos, recuerdo la emoci\'on de haber terminado ese mini ensayo:

\begin{quote}
...“Nos encant\'o hacer el informe, realmente estamos MUY felices por c\'omo qued\'o el informe, por el equipo que hicimos, por la gran ayuda que nos brindaron, por las lindas tardes en el laboratorio, por la frustraci\'on, el amor (y a veces incluso odio) a esos cristales de sulfato de cobre a los cuales vimos \lq\lq{nacer}\rq\rq{} y crecer para salir volando un d\'ia como hoy. ¿Ser\'a casualidad que empezamos a escribir justo el d\'ia en el Cort\'azar habr\'ia cumplido 100 a\~nos? Yo creo en lo que conocemos como destino. Porque comenz\'o como una linda idea para conocer m\'as sobre el trabajo en laboratorio y termin\'o en una extra\~na conexi\'on a la pasi\'on que espero que alguna vez alguien haya sentido por unos cristales de mariposa o unas mariposas cristalizadas (ahora que reflexiono, es lindo pensar que nuestras mariposas simbolicen las miles de mariposas que sobrevolaron Buenos Aires el d\'ia que falleci\'o Julio, como si su homenaje nunca hubiera terminado, sino que qued\'o ah\'i, en la historia: se hizo cristal...). Incluso te digo que queremos a esos cristales, no solo por lo que son, sino por lo que representan”...
\end{quote}

Lo bello de la creatividad humana es que nos permite conectar cosas de modos tan locos. Para m\'i en ese entonces vincular la literatura, con la qu\'imica y el arte fue iluminador. Me despert\'o nuevas rutas neuronales y esas sinapsis no se deshacen tan f\'acilmente...

\begin{quote}
“La qu\'imica es el arte de separar, pesar y distinguir: 
tres ejercicios que tambi\'en resultan \'utiles a quien pretende escribir hechos 
o dar cuerpo a su fantas\'ia” 
\end{quote}

\begin{flushright}
\textit {Primo Levi (L'altrui mestiere, 1985)} 
\end{flushright}

Cort\'azar volver\'ia a cobrar importancia gracias a mi profesora de literatura de 5to, Silvina Marsimi\'an. 

Pensar que por poco me egreso de este colegio sin gusto por el an\'alisis literario. Yo hasta 7mo odiaba leer, me aburr\'ia (en parte porque eso te ense\~na la TV ¿no?). Eso es lo que tristemente piensan muchos, que leer es aburrido. Si bien el colegio me ense\~n\'o a leer, tambi\'en me dej\'o en claro que yo no era para las human\'isticas...llegar al 9/10 todos los a\~nos para no llevarse castellano a diciembre era un acto divino (gracias Jeb\'us). 

Pero con esta profesora pas\'o algo distinto, si bien en la primera prueba cay\'o alto 5, las siguientes notas vinieron de producciones que pude escribir en mi casa, tranquila. ¡Y BUM! Empec\'e a llegar al glorioso 7 (¡incluso al 8!). Resulta ser que padezco de algo a lo cual yo llam\'e \textit {“S\'indrome de la lapicera indecisa”} o \textit {“Bipolar Pen”} (d\'icese de la afecci\'on que impide que una persona escriba sin dudar de cada palabra o signo de puntuaci\'on, de la estructura, coherencia o cohesi\'on del texto producido).

Cuesti\'on que Marsimi\'an nos mand\'o a hacer un ensayo propio sobre alguna tem\'atica y autor en particular. Y ah\'i, voilà, Cort\'azar again. Pero no se trataba de investigar sobre los an\'alisis de textos cr\'iticos hechos sobre una novela en particular de \'el. No. Ac\'a ten\'ia que yo escribir -en base a cierta investigaci\'on a presentar y a partir del trabajo de otros- una hip\'otesis propia, fundamentarla y analizarla. Entender qu\'e trat\'o de decir Julito al escribir Las Armas Secretas, qu\'e mensaje nos quiso transmitir en funci\'on a lo que a \'el le tocaba vivir en aquellos tiempos duros.

Nada.

Para m\'i escribir ese trabajo fue realmente un placer, me hizo ver lo equivocada que estaba sobre la literatura. Prefer\'ia m\'as hacer ese trabajo que hacer las integrales de mi “““gran””” profesor de matem\'atica, Galeano. (¿Yo?...¿una piba de exactas?). Me cay\'o la ficha de que la literatura no solo se trata de estudiar los an\'alisis propuestos por otros, sino que yo tambi\'en puedo crear mis propios an\'alisis y reinterpretarlo todo nuevamente. Fue la primera vez que sent\'i que hice investigaci\'on en serio. 

Nada, tremendo respeto y admiraci\'on a todos los que sigan ese arte. Para m\'i ellos tambi\'en hacen ciencia, hacen investigaci\'on tan v\'alida y verdadera como los de exactas.

Y se terminaba el a\~no y yo re feliz de comenzar mi orientaci\'on en ciencias biol\'ogicas el a\~no siguiente, habiendo sellado una etapa tan fuerte de formaci\'on human\'istica. Finalmente, solo science.

Pero, ¿qu\'e pas\'o con la segunda interrogante? Con los cristales entend\'i que con la ciencia se puede crear arte. Pero hace poco comprend\'i algo nuevo.

La humanidad desde sus principios ha empleado el arte como forma de expresi\'on, de contar qu\'e sent\'ia y viv\'ia, de dejar su marca o al menos una se\~nal pa’los otros...quien quiera que fueran los otros (¿parece que hablo de aliens? Mm mashomenos). Para los nativos de Am\'erica los europeos eran de re otro mundo, o m\'as bien de otra galaxia. Pero bueno, bien sabemos que los europeos trataron de evangelizar a los americanos mostr\'andoles lo divino de Dios y sus poderes, para lo cual ¿qu\'e emplearon? S\'i, el arte. Pinturas y pinturas, hechas ac\'a con lo que hab\'ia. Cubrieron las paredes de todas las miles y miles de iglesias que se construyeron en toda la Patria Grande, sobre todo, en el camino de las Rutas del Plata y Oro.

Aj\'a, cute lecci\'on de historia pero ¿qu\'e tiene que ver esto con todo lo otro? Qu’en Exactas resulta que se estudia la composici\'on qu\'imica de varias pinturas provenientes de ciertas iglesias y que por alguna raz\'on de la vida yo qued\'e metida en ese lugar. Como dije antes, yo creo en el destino (o al menos tengo arraigado muy fuertemente la afinidad humana por la causalidad) y parece que este me ha jugado otra buena pasada. 

De todos los millares de iglesias, de todas aquellas que fueron saqueadas y destruidas, y de las que no, de todos los millares de cuadros que se hicieron en aquellos tiempos, de aquellos que quedaron y de los que no qued\'o nada, de todos los grupos y laboratorios que estudian arqueolog\'ia, yo termin\'e en aquel que estudi\'o pinturas del mismo pueblo en el que naci\'o mi viejo, mis abuelos y mis bisabuelos. Termin\'e en el lugar en donde se estudi\'o esa pintura que muy probablemente alg\'un antepasado m\'io cercano haya mirado con cara de WTF, “no entendo qu\'e quieren estos blancos”. Esa clase de pueblos que nadie conoce, cuyos nombres solo los hijos de los hijos habr\'an escuchado alguna vez, ese pueblo en el medio de la nada y de las monta\~nas en el que se da todav\'ia hoy ofrendas a la Madre Tierra (y ojal\'a que jam\'as se deje de hacer), ese nombre en Aymar\'a que nada tiene que ver con la qu\'imica que yo vengo aprendiendo en estos a\~nos en el Colegio de la Reina del Plata, ni con Siri y su \textit {“¡¡Golazo!!”}, ese que se contrapone rotundamente a este mundo bien Europeo del CNBA... y voilà. Ese mundo del m\'as profundo origen de mi ser se conect\'o a este mundo que vivo hoy ac\'a.

¿Qu\'e es de la ciencia en el arte? La ciencia en el arte te puede conectar mundos opuestos, galaxias distintas (y si nos ponemos bien metaf\'oricos puedo decir que fue el agujero negro gusano entre dos mundos paralelos). A m\'i me conect\'o esos dos mundos que viven en m\'i...me los hizo uno. Me hizo una.

Se dice que human\'isticas o ciencia, que lo art\'istico o lo acad\'emico, as\'i como se dice Am\'erica o Europa. Pero yo me siento todo eso al mismo tiempo. 

\begin{quote}“Quer\'ia transmitir una emoci\'on que tuve acerca de la belleza del universo. Es dif\'icil de describir porque es una emoci\'on. Es an\'alogo al sentimiento que uno tiene en la religi\'on sobre la existencia de un Dios que lo controla todo en el universo: hay una apreciaci\'on general que vos ten\'es cuando pens\'as sobre c\'omo las cosas que parecieran ser tan distintas y comportarse tan distinto, est\'an manejadas desde “atr\'as de escena” por la misma organizaci\'on, la mismas leyes f\'isicas. Es una apreciaci\'on de la belleza matem\'atica del universo, de c\'omo funciona desde adentro; comprender que el fen\'omeno que vemos resulta venir de la complejidad del funcionamiento interno entre los \'atomos, un sentimiento de qu\'e tan dram\'atico y maravilloso es. Es el sentimiento de la maravilla –de una maravilla cientifica- la cual yo sent\'i que pod\'ia ser transmitida a trav\'es de un dibujo a alguien que tambi\'en haya tenido esa emoci\'on. Podr\'ia recordarle, por un momento, de este sentimiento sobre la gloria del universo.” 
\end{quote}

\begin{flushright}
\textit {Richard Feymann (Surely you’re joking me Mr. Feymann, 1985)}
\end{flushright}

\textbf{Ricky, yo a trav\'es de ese cristal y ese trozo de pintura pude sentir ese awe.}

\textbf{Y fue hermoso.}\\

\textit{(Eso o el dietil \'eter me est\'a pegando duro).} \\

Pd. Comparto esto porque lo que uno no comparte con otros...muere en uno. Y no quiero escuchar a nadie hablar de human\'isticas o exactas o biol\'ogicas o art\'isticas...y s\'i un todo con todo \small carajo


\end{document}
