\documentclass[12t, a4 paper]{article}


\usepackage{lmodern}
\usepackage[T1]{fontenc}
\usepackage[utf8]{inputenc} 
\usepackage[spanish]{babel}



\title{Y estaré esperándolos  }
\date{}

\begin{document}

\maketitle

\textbf{Previa de cualquier despedida...} \\

El tiempo pasa, y así como fueron llegando a mi vida, se van yendo. Sé que de esto se trata la vida, los amigos a la larga van y vienen. Hablando una vez con Charlotte y Mary, me dijeron que eso depende de qué tipo de relaciones arme uno...En este primer año y medio de vida real creo haber establecido un lazo muy especial con gente muy diversa, un lazo que no está basado en conversaciones diarias sino en un intercambio de pensamientos...verdaderos. Por primera vez siento que me compartí realmente, que fui yo con otros. No podría estar más feliz de haber conocido gente así. Sin embargo, ahora mismo las lágrimas inundan mis párpados. ¿Qué sucede? 

Pasar de una etapa a otra es doloroso. Al menos para mí separarme de la gente con la que pude establecer ese lazo me duele muchísimo. Quise madurar en este año y medio, quise crecer, pero parece que no ha sido suficiente. Nunca es suficiente. 

Terminar sexto es raro. Se hace tangible que el tiempo y espacio nos separa, se hace tangible la brecha que la vida traza entre las personas. Todos lo sabemos. Sabemos que con el tiempo dejaremos de vernos y que esas dichos futuros encuentros no se concretarán. Tal vez lo duro ahora sea la etapa intermedia: aquella en la que ves cómo se van...Si tan solo fuera rápido no dolería tanto. Sería un día de gris y luego, back to real life como after camp. Pero no. Aquí estamos Belu, lidiando con esto. Entonces buscamos una solución, algo que lo haga más fácil. Soy rara, lo sé. A veces me gustaría no ser tú Belu, me gustaría ser otra persona para saber qué siente, cómo lo ve...antes yo era esa persona. Y nada, simplemente seguía viviendo...¿será que aquellos lazos no eran tan especiales como los de ahora? Así lo creo pero..¡¿cómo saberlo?! Espero no ser la única...espero poder animarme a sacarme esto y compartirlo, porque de lo contrario va a morir en mí. (Escribo escribo pero todavía duele). Por momentos siento que siendo fría y cortante hará que se vayan y ya, pero al rato siento que fui una tonta. Como bien escribí en otra ocasión, no todos tenemos la dicha de tener lo que tenemos. Estar agradecido por ello y demostrarlo es lo menos que podemos hacer. Yo me siento muy agradecida por tener las amistades que he logrado, las palabras y charlas que he tenido...incluso las lágrimas que me han hecho derramar. Entonces, ¿estoy haciendo bien? ¿Estoy haciendo bien siendo fría? Temo. Temo no llegar de decirles gracias por ello. Temo dejar de hablarles como me pasó con otros. Temo no llegar a decírselos, temo que la brecha ya sea lo suficientemente grande como para compartirles esa última charla. Temo que llegue ese momento, temo no saber cuál es el momento. Está cerca, lo sé. De varios está cerca y acá estoy. Temiendo. 

Belu, si te vas a quedar ahí temiendo...el poco tiempo que te queda con ellos va a ser desperdiciado. No temas. Andá y sonreíles, así como ellos te devolvieron la sonrisa. Pero se van a ir, ¿y? Eso nos pone mal, sí lo sé. Pero no hay nada que pueda detener eso. No hay solución. Tendrás que llorar y secar las lagrimas rápido. No. Eso es lo que sos. Cuando sonreís, sos Belu. Cuando te enojás, sos Belu. Y cuando llorás, también sos Belu.  

Así como han llegado se han ido, y solo espero que algún día, en algún momento, así como llegaron, vuelvan a aparecer. Y estaré esperándolos. 
\end{document}